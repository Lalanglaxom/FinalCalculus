\documentclass[12pt,a4paper]{article}
\usepackage[utf8]{inputenc}
\usepackage[vietnamese]{babel}
\usepackage{amsmath}
\usepackage{amssymb}
\usepackage{amsfonts}
\usepackage{geometry}
\usepackage{array}
\usepackage{booktabs}

\geometry{margin=2.5cm}

\title{\textbf{BÀI TẬP GIẢI TÍCH}}
\author{Họ và tên: .............................}
\date{\today}

\begin{document}

\maketitle

%========================================
% CÂU 3
%========================================
\section*{Câu 3. Tìm đạo hàm $\dfrac{dy}{dx}$ của các hàm số sau (1.0 điểm)}

\textbf{Các công thức cần nhớ:}
\begin{itemize}
    \item Đạo hàm căn bậc hai: $(\sqrt{x})' = \dfrac{1}{2\sqrt{x}}$
    \item Đạo hàm thương: $\left(\dfrac{u}{v}\right)' = \dfrac{u'v - uv'}{v^2}$
    \item Đạo hàm hàm hợp: $(u^n)' = n \cdot u^{n-1} \cdot u'$
\end{itemize}

\bigskip

%----------------------------------------
% Câu 3a
%----------------------------------------
\subsection*{a) $y = \dfrac{\sqrt{x}-4}{\sqrt{x}+4}$}

\textbf{Giải:}

Đặt $u = \sqrt{x} - 4$ và $v = \sqrt{x} + 4$

Ta có:
\begin{align*}
    u' &= (\sqrt{x} - 4)' = \frac{1}{2\sqrt{x}} \\[8pt]
    v' &= (\sqrt{x} + 4)' = \frac{1}{2\sqrt{x}}
\end{align*}

Áp dụng công thức đạo hàm của thương: $\left(\dfrac{u}{v}\right)' = \dfrac{u'v - uv'}{v^2}$

\begin{align*}
    \frac{dy}{dx} &= \frac{u'v - uv'}{v^2} \\[10pt]
    &= \frac{\dfrac{1}{2\sqrt{x}} \cdot (\sqrt{x}+4) - (\sqrt{x}-4) \cdot \dfrac{1}{2\sqrt{x}}}{(\sqrt{x}+4)^2} \\[10pt]
    &= \frac{\dfrac{1}{2\sqrt{x}} \left[(\sqrt{x}+4) - (\sqrt{x}-4)\right]}{(\sqrt{x}+4)^2} \\[10pt]
    &= \frac{\dfrac{1}{2\sqrt{x}} \cdot (\sqrt{x}+4 - \sqrt{x}+4)}{(\sqrt{x}+4)^2} \\[10pt]
    &= \frac{\dfrac{1}{2\sqrt{x}} \cdot 8}{(\sqrt{x}+4)^2} \\[10pt]
    &= \frac{8}{2\sqrt{x} \cdot (\sqrt{x}+4)^2} \\[10pt]
    &= \frac{4}{\sqrt{x} \cdot (\sqrt{x}+4)^2}
\end{align*}

\textbf{Kết luận:} $\dfrac{dy}{dx} = \boxed{\dfrac{4}{\sqrt{x}(\sqrt{x}+4)^2}}$

\bigskip

%----------------------------------------
% Câu 3b
%----------------------------------------
\subsection*{b) $y = \left(\dfrac{\sqrt{x}}{10} - 1\right)^{-10}$}

\textbf{Giải:}

Đặt $u = \dfrac{\sqrt{x}}{10} - 1$, ta có $y = u^{-10}$

\textbf{Bước 1:} Tính $u'$
\begin{align*}
    u &= \frac{\sqrt{x}}{10} - 1 = \frac{1}{10}\sqrt{x} - 1 \\[8pt]
    u' &= \frac{1}{10} \cdot \frac{1}{2\sqrt{x}} = \frac{1}{20\sqrt{x}}
\end{align*}

\textbf{Bước 2:} Áp dụng công thức đạo hàm hàm hợp: $(u^n)' = n \cdot u^{n-1} \cdot u'$

\begin{align*}
    \frac{dy}{dx} &= -10 \cdot u^{-11} \cdot u' \\[10pt]
    &= -10 \cdot \left(\frac{\sqrt{x}}{10} - 1\right)^{-11} \cdot \frac{1}{20\sqrt{x}} \\[10pt]
    &= \frac{-10}{20\sqrt{x}} \cdot \left(\frac{\sqrt{x}}{10} - 1\right)^{-11} \\[10pt]
    &= \frac{-1}{2\sqrt{x}} \cdot \left(\frac{\sqrt{x}}{10} - 1\right)^{-11}
\end{align*}

\textbf{Cách viết khác:}
\begin{align*}
    \frac{dy}{dx} &= \frac{-1}{2\sqrt{x} \cdot \left(\dfrac{\sqrt{x}}{10} - 1\right)^{11}} \\[10pt]
    &= \frac{-1}{2\sqrt{x} \cdot \left(\dfrac{\sqrt{x} - 10}{10}\right)^{11}} \\[10pt]
    &= \frac{-10^{11}}{2\sqrt{x} \cdot (\sqrt{x} - 10)^{11}}
\end{align*}

\textbf{Kết luận:} $\dfrac{dy}{dx} = \boxed{\dfrac{-1}{2\sqrt{x}\left(\dfrac{\sqrt{x}}{10} - 1\right)^{11}}}$

\bigskip

%----------------------------------------
% Bảng tóm tắt câu 3
%----------------------------------------
\subsection*{Bảng tóm tắt kết quả Câu 3:}

\begin{center}
\renewcommand{\arraystretch}{2.5}
\begin{tabular}{|c|c|}
\hline
\textbf{Hàm số} & \textbf{Đạo hàm $\dfrac{dy}{dx}$} \\
\hline
$y = \dfrac{\sqrt{x}-4}{\sqrt{x}+4}$ & $\dfrac{4}{\sqrt{x}(\sqrt{x}+4)^2}$ \\
\hline
$y = \left(\dfrac{\sqrt{x}}{10} - 1\right)^{-10}$ & $\dfrac{-1}{2\sqrt{x}\left(\dfrac{\sqrt{x}}{10} - 1\right)^{11}}$ \\
\hline
\end{tabular}
\end{center}

\newpage

%========================================
% CÂU 4
%========================================
\section*{Câu 4. Tìm phương trình tiếp tuyến của đồ thị $y = 1 + 2e^x$ tại điểm có $x = 0$ (1.0 điểm)}

\textbf{Công thức phương trình tiếp tuyến:}

Phương trình tiếp tuyến của đồ thị hàm số $y = f(x)$ tại điểm $M(x_0, y_0)$:
$$y - y_0 = f'(x_0)(x - x_0)$$

hoặc viết dưới dạng:
$$y = f'(x_0)(x - x_0) + y_0$$

\bigskip

\textbf{Giải:}

Cho hàm số $y = 1 + 2e^x$ và $x_0 = 0$

\bigskip

\textbf{Bước 1:} Tính tọa độ tiếp điểm

Tại $x_0 = 0$:
\begin{align*}
    y_0 &= 1 + 2e^0 \\
    &= 1 + 2 \cdot 1 \\
    &= 1 + 2 \\
    &= 3
\end{align*}

Vậy tiếp điểm là $M(0, 3)$.

\bigskip

\textbf{Bước 2:} Tính đạo hàm $y'$

\begin{align*}
    y &= 1 + 2e^x \\
    y' &= (1)' + (2e^x)' \\
    &= 0 + 2e^x \\
    &= 2e^x
\end{align*}

\bigskip

\textbf{Bước 3:} Tính hệ số góc của tiếp tuyến

Hệ số góc của tiếp tuyến tại $x_0 = 0$:
\begin{align*}
    k = y'(0) &= 2e^0 \\
    &= 2 \cdot 1 \\
    &= 2
\end{align*}

\bigskip

\textbf{Bước 4:} Viết phương trình tiếp tuyến

Áp dụng công thức phương trình tiếp tuyến:
\begin{align*}
    y - y_0 &= k(x - x_0) \\
    y - 3 &= 2(x - 0) \\
    y - 3 &= 2x \\
    y &= 2x + 3
\end{align*}

\bigskip

\textbf{Kết luận:} Phương trình tiếp tuyến của đồ thị $y = 1 + 2e^x$ tại điểm có $x = 0$ là:
$$\boxed{y = 2x + 3}$$

\bigskip

%----------------------------------------
% Bảng tóm tắt câu 4
%----------------------------------------
\subsection*{Bảng tóm tắt kết quả Câu 4:}

\begin{center}
\renewcommand{\arraystretch}{1.8}
\begin{tabular}{|l|c|}
\hline
\textbf{Thông tin} & \textbf{Giá trị} \\
\hline
Hàm số & $y = 1 + 2e^x$ \\
\hline
Hoành độ tiếp điểm & $x_0 = 0$ \\
\hline
Tung độ tiếp điểm & $y_0 = 3$ \\
\hline
Tiếp điểm & $M(0, 3)$ \\
\hline
Đạo hàm & $y' = 2e^x$ \\
\hline
Hệ số góc tiếp tuyến & $k = y'(0) = 2$ \\
\hline
\textbf{Phương trình tiếp tuyến} & $\mathbf{y = 2x + 3}$ \\
\hline
\end{tabular}
\end{center}

\bigskip

\textbf{Nhận xét:}
\begin{itemize}
    \item Tiếp tuyến có hệ số góc $k = 2 > 0$ nên tiếp tuyến là đường thẳng đi lên từ trái sang phải.
    \item Tiếp tuyến cắt trục $Oy$ tại điểm $(0, 3)$ chính là tiếp điểm.
    \item Tiếp tuyến cắt trục $Ox$ tại điểm $\left(-\dfrac{3}{2}, 0\right)$.
\end{itemize}

\end{document}